% LaTeX Article Template - customizing page format
%
% LaTeX document uses 10-point fonts by default.  To use
% 11-point or 12-point fonts, use \documentclass[11pt]{article}
% or \documentclass[12pt]{article}.
\documentclass{article}

% Set left margin - The default is 1 inch, so the following 
% command sets a 1.25-inch left margin.
\setlength{\oddsidemargin}{0.25in}

% Set width of the text - What is left will be the right margin.
% In this case, right margin is 8.5in - 1.25in - 6in = 1.25in.
\setlength{\textwidth}{6in}

% Set top margin - The default is 1 inch, so the following 
% command sets a 0.75-inch top margin.
\setlength{\topmargin}{-0.25in}

% Set height of the text - What is left will be the bottom margin.
% In this case, bottom margin is 11in - 0.75in - 9.5in = 0.75in
\setlength{\textheight}{8in}
\usepackage{fancyhdr}
\usepackage{float}
\usepackage{mathtools}
\usepackage{amsmath}
\usepackage{amssymb}
\usepackage{graphicx}
\usepackage{float}
\usepackage{tikz}
\usetikzlibrary{positioning}
\graphicspath{ {./} }
\setlength{\parskip}{5pt} 
\pagestyle{fancyplain}

% Universes
\newcommand{\NN}{\mathbb{N}}
\newcommand{\ZZ}{\mathbb{Z}}
\newcommand{\QQ}{\mathbb{Q}}
\newcommand{\RR}{\mathbb{R}}
\newcommand{\CC}{\mathbb{C}}

% Groups commands
\newcommand{\inv}{^{-1}}
\newcommand{\lcm}{\mathrm{lcm}}
\newcommand{\lr}[1]{\langle #1 \rangle}
% Set the beginning of a LaTeX document
\begin{document}

\lhead{Drew Remmenga MATH 458}
\rhead{HW \#5}
%\lhead{Independent Study}
%\rhead{R Lab}


\begin{enumerate}
\item (Problem from the Reading Quiz) 
    
\textbf{Lagrange's Theorem.} Let $G$ be a finite group. If $H$ is a subgroup of $G$, then $|H|$ divides $|G|$.

The following example in the textbook is given to show the converse of Lagrange's Theorem is false. 
The author left out a lot of the details in the example (as usual). Fill in the missing details to write a more complete explanation. 
\begin{enumerate}
\item State the converse of Lagrange's Theorem. Given a group $G$ for any non empty subset $H$ of $G$ if $|H|$ divides $|G|$ then $H$ is a subgroup of $G$.

\item $A_4$ has eight elements of order 3 ($\alpha_5$ through $\alpha_{12}$ in the Table 5.1 in text).
Explain why those elements and no others have order 3. (Do not compute the orders directly.)  $A_4 = \{I,(1,2,3),(1,3,2),(1,2,4),(1,4,2),(1,3,4),(1,4,3),(2,3,4),(2,4,3),(1,2)(3,4),(1,3)(2,4),(1,4)(2,3)\}$ Clearly $A_4$ has 8 elements of three cycles so we can directly say it has 8 elements of order 3.

\item Define the notation used.
\begin{enumerate}
    
    \item Let $H$ be a subgroup of $A_4$ of order 6. 
    
    \item Let $a$ be an element of order 3 in $A_4$. 

\end{enumerate}

\item Suppose $a\not\in H$. Then $A_4=H\cup aH$. Why? Given that $A \not \in H$. Then to show that $A_4 = H\cup aH$. As $a$ be an element of order 3 $a \in A_4$. Then $H$ be a subgroup of $A_4$ So $e \in H$ so $ae \in a H$. Then as $a \not \in H$ but $a \in aH \implies a \in H \cup aH$. Now let $x \in H \cup aH \implies $either $x \in H$ or $x\in aH$. If $x \in H$ and $H$ is a subgroup of $A_4$ order 6. Clearly $H \subseteq A_4$ due to subgroup definition. So $x \in A_4 \implies H \cup aH \subseteq A_4$. This implies $A_4 = H \cup aH$.

\item Then $a^2\in H$ or $a^2\in aH$. Why? As $a \in A_4 \implies a^{3} = e$ because it is element of order 3 in $A_4$. And also given $H$ is a subgroup of order 6 it means it has at least one lement of order 6. Now as $a^{3} = e \implies (a^{3})^{2} = a^{6} = e^{2} =e$. Thus $a^{2} \in H$. Or if $a^{2} \not \in H$ then $a^{2} \in aH$ is trivially true. 

\item If $a^2\in H$, then $a\in H$. Why? And why is ``this case ruled out''? If given $a^{2} \in H$. Then as $H$ is a subgroup of $A_4$ of order 6 so $(a^{2})^{6} = e = a^{12}$. By right cancellation law we obtain $a^{3} = e \implies a \in H$. Now this is ruled out because it is given that $a \not \in H$.

\item If $a^2\in aH$, then $a^2=ah$ for some $h\in H$. So $a\in H$. Why? For $a^{2} \in a H \implies a^2 = ah $ for some $h \in H$. Now by left cancellation law $a^{2}=aa=ah$ Multiply by $a^{-1}$ on both sides and we get $a=h$. So $a \in H$. 

\item So any subgroup of $A_4$ of order 6 size must contain all elements of $A_4$ of order 3. Why? And why is this ``absurd''? As given $H$ is a subgroup of order 6. That is it has only 6 elements in numbers how is possible that all the eight elements of order 3 within itself. So this statement is absurd in that it should contain all eight elements of order 3. 

\item What initial assumption must be false given this contradiction? The inital statement let $H$ be a subgroup of $A_4$ of order 6 is contradictory. 

\item How does this example show the converse of Lagrange's Theorem is false (refer to your answer in part (a))? Now as the order of $A_4$ is 12 and we also have taken the subset of given group of order 6. 6 divides 12 but it is not true that the subset becomes a subgroup. 

\end{enumerate}


\item Let $H$ be a subgroup of the $D_n$ with odd order. Show that $H$ must be cyclic. (Recall: $D_n$ can be defined as $D_n=\langle r,f\rangle$ where $|r|=n$, $|f|=2$, and $r^kf=fr^{-k}$.) Let $H$ denote reflections. Then we have $a = r^{i} f$ and $b = r^{j} f$ both $\in H$. Then $r^{i}f(r^{j}f)^{-1} = (r^{i}f)^{-1} r ^{j} f = fr^{-i} r ^{j} = f^{2} r ^{i -j}$. Then we have $r^{i-j} \in H$ which is a contradiction. Hence the set of reflections can not form a subgroup. Hence $H$ must be with only rotations. So $|H|$ = n = odd and $H = \langle r^{k} \rangle$. Where $k \| n$. As set of rotations forms a group and $r^{n} = e$ and $r^{i} r{j} - r ^{j} r^{i}$ ie comutative. Hence $H$ is a cyclic group of odd order generated by $r$.

\item Let $H$ be a subgroup of $S_n$. Use properties of cosets to prove that either every member of $H$ is an even permutation or exactly half of the members is even. Let $H$ be a subgroup of $S_n$. If $H$ contains no odd permutations, then $H$ contains only even permutations, and we're done. Otherwise, let $o\in H$ be an odd permutation and consider the function $f:H\rightarrow H$ that multiplies each element by $o$. Note that this function is bijective: it's injective, with an inverse function that multiplies each element by $o^{-1}$, and it's surjective, because for every element $h \in H$, we can find an element $h\dot o^{-1}$ that $f$ maps into it. Note that multiplying by an odd permutation changes odd permutations into even permutations and vice versa. It follows that f is a bijection that perfectly pairs up the odd permutations in H with the even permutations. Hence there are exactly as many odd permutations as even permutations in H.

\item Prove that an abelian group of order 15 is cyclic. Do not use Cauchy's Theorem for Abelian Groups. Let $G$ be a group of order 15. Since 15 is divisible by two prime numbers by the theoreom of abelian groups of semi prime order being cyclic all abelian groups of order 15 are cyclic. 

\end{enumerate}
\end{document}