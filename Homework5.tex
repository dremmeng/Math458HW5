\documentclass{article}

\usepackage[margin=1in]{geometry}
\usepackage{fancyhdr, lastpage}
\usepackage{tikz}
\usepackage{amsmath,amssymb,amsthm}
\usetikzlibrary{calc}


% Universes
\newcommand{\NN}{\mathbb{N}}
\newcommand{\ZZ}{\mathbb{Z}}
\newcommand{\QQ}{\mathbb{Q}}
\newcommand{\RR}{\mathbb{R}}
\newcommand{\CC}{\mathbb{C}}

% Groups commands
\newcommand{\inv}{^{-1}}
\newcommand{\lcm}{\mathrm{lcm}}
\newcommand{\lr}[1]{\langle #1 \rangle}
\newcommand{\Inn}{\mathrm{Inn}}
\newcommand{\iso}{\cong}

%%%%%%%%%%%%%%%%%%%%%%%%%%%%%%%%%%%%%%%%%%%%%%%%%%%%%%%%%%%%%%
\setlength{\parindent}{0cm}
\pagestyle{fancy}
\lhead{MATH458 Abstract Algebra}
\rhead{Homework 4}

%%%%%%%%%%%%%%%%%%%%%%%%%%%%%%%%%%%%%%%%%%%%%%%%%%%%%%%%%%%%%%
\begin{document}
\section*{Homework 4}


\begin{enumerate}
\item (Problem from the Reading Quiz) 
    
\textbf{Lagrange's Theorem.} Let $G$ be a finite group. If $H$ is a subgroup of $G$, then $|H|$ divides $|G|$.

%%% Delete lines 42-45 since you don't have the image file I used
The following example in the textbook is given to show the converse of Lagrange's Theorem is false. 
\begin{center}
\includegraphics[scale=.45]{LagrangeConverse.png}
\end{center}
\vspace{-1em}
The author left out a lot of the details in the example (as usual). Fill in the missing details to write a more complete explanation. 
\begin{enumerate}
\item State the converse of Lagrange's Theorem.

\item $A_4$ has eight elements of order 3 ($\alpha_5$ through $\alpha_{12}$ in the Table 5.1 in text).
Explain why those elements and no others have order 3. (Do not compute the orders directly.) 

\item Define the notation used.
\begin{enumerate}
    
    \item Let $H$ be \dots 
    
    \item Let $a$ be \dots 

\end{enumerate}

\item Suppose $a\not\in H$. Then $A_4=H\cup aH$. Why?

\item Then $a^2\in H$ or $a^2\in aH$. Why?

\item If $a^2\in H$, then $a\in H$. Why? And why is ``this case ruled out''?

\item If $a^2\in aH$, then $a^2=ah$ for some $h\in H$. So $a\in H$. Why?

\item So any subgroup of $A_4$ of order 6 size must contain all elements of $A_4$ of order 3. Why? And why is this ``absurd''?

\item What initial assumption must be false given this contradiction?

\item How does this example show the converse of Lagrange's Theorem is false (refer to your answer in part (a))?

\end{enumerate}


\item Let $H$ be a subgroup of the $D_n$ with odd order. Show that $H$ must be cyclic. (Recall: $D_n$ can be defined as $D_n=\langle r,f\rangle$ where $|r|=n$, $|f|=2$, and $r^kf=fr^{-k}$.)

\item Let $H$ be a subgroup of $S_n$. Use properties of cosets to prove that either every member of $H$ is an even permutation or exactly half of the members is even. 

\item Prove that an abelian group of order 15 is cyclic. Do not use Cauchy's Theorem for Abelian Groups. 

\end{enumerate}
\end{document}